\documentclass[11pt]{article} 

\usepackage[utf8]{inputenc} 

\usepackage{graphicx}

\usepackage{geometry}
\geometry{a4paper} 


\title{Winterthur Yard: Abschlussbericht Projekt}
\author{Maja Fritschi, Raphael Spörri, Florian Bosshard}
\date{} 
\begin{document}
\maketitle

\tableofcontents
\newpage

\section{Projektteam}
Das Projekt wird von folgenden Personen entwickelt:
\begin{itemize}
\item Maja Fritschi (fritsmaj)
\item Raphael Spörri (sporrra0)
\item Florian Bosshard (bosshflo)
\end{itemize}


\section{Umgesetzte Funktionalitäten}
Folgend werden Funktionalitäten aufgelistet die bis zum18. Dezember 2013 umgesetzt wurden: 
\begin{itemize}
\item Der Spieler kann sich auf der Startseite mit einem Usernamen einloggen. Geprüft wird nur, ob er etwas eingibt. Kein Username ist nicht erlaubt. Ausserdem kann man auch nicht auf eine Subseite zugreiffen, wenn man sich nicht eingeloggt hat.
\item Hat der Spieler sich erfolgreich eingeloggt, sieht er einen  Kartenausschnitt der Winterthureraltstatt. Weiter wird ihm mit einem roten Icon angezeigt, wo er sich befindet. Andere Spieler die ebenfalls angemeldet sind werden mit blauen Icons angezeigt. Der MisterX wird zu beginn noch nicht angezeigt.
\item Befindet sich der Spieler aufeinem der eingezeichneten Koordinaten-Punkte, kann er den Button ``MisterX fangen'' drücken. Es wird Ihm im Erfogsfall gesagt, dass er den MisterX gefangen hat.  Ansonsten erfährt er wo sich der MisterX zu einem bestimmten Zeitpunkt befunden hat. Der ehemalige Standpunkt vom MisterX wird nun auch auf der Karte eingezeichnet. 
\end{itemize}

Die aktuelle Applikation kann unter http://yard.prusik.ch angeschaut werden. 

\section{Aufgetretene Probleme}
Wir haben anfangs versucht mit jQueryMobile 1.4 zu arbeiten, sind dann aber auf Probleme beim Einbetten der Karte gestossen. Daher haben wir zurück auf jQueryMobile 1.3 gewechselt.

Beim Loginscreen erscheint in gewissen Browsern kein Scrolling. Das Bild oberhalb des Login-Formulars skaliert nicht korrekt. 



\section{Screenshots}





\end{document}
